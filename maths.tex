%%%%%%%%%%%%%%%%%%%%%%%%%%%%%%%%%%%%%%%%%%%%%%%%%%%%%%%%%%%%%%%%%%%%%%%%%%%%%%%
%
% magnolia-maths.cls, LaTeX class for the free teaching project Velvia
% Copyright (C) 2004-2006 Francois Fayard <fayard.prof@gmail.com>.
% 
% This program is free software; you can redistribute it and/or
% modify it under the terms of the GNU General Public License
% as published by the Free Software Foundation; either version 2
% of the License, or (at your option) any later version.
% 
% This program is distributed in the hope that it will be useful,
% but WITHOUT ANY WARRANTY; without even the implied warranty of
% MERCHANTABILITY or FITNESS FOR A PARTICULAR PURPOSE.  See the
% GNU General Public License for more details.
% 
% You should have received a copy of the GNU General Public License
% along with this program; if not, write theo the Free Software Foundation
% Inc., 51 Franklin Street, Fifth Floor, Boston, MA  02110-1301, USA.
%
%%%%%%%%%%%%%%%%%%%%%%%%%%%%%%%%%%%%%%%%%%%%%%%%%%%%%%%%%%%%%%%%%%%%%%%%%%%%%%%
%
% PLEASE DO NOT MAKE ANY CHANGES TO THIS FILE!
% 
% If you wish to make changes to this file, rename this file
% to something other than exam.cls BEFORE YOU MAKE THE CHANGES!
% 
% If there's some feature that you'd like that this file doesn't
% provide, tell me about it.
%
%%%%%%%%%%%%%%%%%%%%%%%%%%%%%%%%%%%%%%%%%%%%%%%%%%%%%%%%%%%%%%%%%%%%%%%%%%%%%%%
%
% On charge les macros mathématiques de l'ams
\usepackage{amsmath,amssymb,amsthm,stmaryrd,mathtools}
\usepackage{mathrsfs}


\usepackage[french]{babel}
\usepackage{amsthm}
\usepackage{thmtools}
\usepackage{amsmath}
\usepackage{subcaption}
\usepackage{graphicx}
\usepackage{amsfonts}
\usepackage{float}
\usepackage{mathabx}
\usepackage{epigraph} 
\usepackage{varwidth}
\usepackage{indentfirst}
\usepackage{fancyhdr}
\usepackage{lastpage}
\usepackage{minitoc}
\usepackage{pstcol}
\usepackage{multicol}
\usepackage{enumitem}
\usepackage{multirow}
\usepackage[table]{xcolor}
\usepackage{color}
\definecolor{gr1}{rgb}{0.94,0.97,1.0}
\usepackage{pgfplots}
\pgfplotsset{compat=1.17}

\declaretheoremstyle[
    spaceabove=-6pt, 
    spacebelow=6pt, 
    headfont=\normalfont\bfseries, 
    bodyfont = \normalfont,
    postheadspace=1em,  
    headpunct={ :}]{style_demo} %<---- change this name
\declaretheorem[name={Démonstration}, style=style_demo, unnumbered]{demo}

\declaretheorem[name=Théorème, thmbox=S]{theorem}
\declaretheorem[name=Lemme, unnumbered]{lemme}
\declaretheorem[name=Corollaire, thmbox=S, sibling = theorem]{corr}
\declaretheorem[name=Propriété, thmbox=S]{prop}
% \declaretheorem[name=Proprietatea]{prop}

\declaretheorem[name=Définition, thmbox=S]{defn}
\declaretheorem[name=Remarque]{rmq}
\declaretheorem[name=Application]{appcrs}
\declaretheorem[name=Exemple]{exmp}
\declaretheorem[name=Coin culture, thmbox=L]{ccult}
\declaretheorem[name=Rappels]{rap}
\declaretheorem[name=Vocabulaire]{vocab}
\declaretheorem[name=Notation]{notation}

%
%\ifthenelse{\equal{\mag@tex@minimal}{oui}}{}{\usepackage{stmaryrd}}

%%%%%%%%%%%%%%%%%%%%%%%%%%%%%%%%%%%%%%%%%%%%%%%%%%%%%%%%%%%%%%%%%%%%%%%%%%%%%%%
%                                                                             %
%                                                                             %
%                               Ajouts                                        %
%                                                                             %
%                                                                             %
%%%%%%%%%%%%%%%%%%%%%%%%%%%%%%%%%%%%%%%%%%%%%%%%%%%%%%%%%%%%%%%%%%%%%%%%%%%%%%%

\DeclareMathOperator{\sinc}{sinc}
\newcommand{\vect}[1]{\overrightarrow{#1}}

\DeclareMathOperator{\dt}{\mathrm{d} t}
\renewcommand{\d}[1]{\mathrm{d\left(#1 \right)}}

\DeclareMathOperator{\intainf}{\int_{a}^{+\infty}}
\DeclareMathOperator{\intxinf}{\int_{x}^{+\infty}}
\DeclareMathOperator{\intoinf}{\int_{0}^{+\infty}}
\DeclareMathOperator{\intax}{\int_{a}^{x}}
\DeclareMathOperator{\intab}{\int_{a}^{b}}
\DeclareMathOperator{\intI}{\int_{I}}

\DeclareMathOperator{\sumon}{\sum_{k=0}^{n}}
\DeclareMathOperator{\sumonI}{\sum_{i\in I}}
\DeclareMathOperator{\sumonK}{\sum_{k\in K}}
\DeclareMathOperator{\sumonN}{\sum_{n\in \N}}
\DeclareMathOperator{\sumonNs}{\sum_{n\in \Ns}}
\DeclareMathOperator{\sumonJ}{\sum_{j\in J}}

\DeclareMathOperator{\prodon}{\prod_{k=0}^{n}}
\DeclareMathOperator{\prodonI}{\prod_{i\in I}}
\DeclareMathOperator{\prodonK}{\prod_{k\in K}}
\DeclareMathOperator{\prodonN}{\prod_{n\in \N}}
\DeclareMathOperator{\prodonNs}{\prod_{n\in \Ns}}
\DeclareMathOperator{\prodonJ}{\prod_{j\in J}}

\DeclareMathOperator{\bigcupon}{\bigcup_{k=0}^{n}}
\DeclareMathOperator{\bigcuponI}{\bigcup_{i\in I}}
\DeclareMathOperator{\bigcuponK}{\bigcup_{k\in K}}
\DeclareMathOperator{\bigcuponN}{\bigcup_{n\in \N}}
\DeclareMathOperator{\bigcuponNs}{\bigcup_{n\in \Ns}}
\DeclareMathOperator{\bigcuponJ}{\bigcup_{j\in J}}

\DeclareMathOperator{\bigsqcupon}{\bigsqcup_{k=0}^{n}}
\DeclareMathOperator{\bigsqcuponI}{\bigsqcup_{i\in I}}
\DeclareMathOperator{\bigsqcuponK}{\bigsqcup_{k\in K}}
\DeclareMathOperator{\bigsqcuponN}{\bigsqcup_{n\in \N}}
\DeclareMathOperator{\bigsqcuponNs}{\bigsqcup_{n\in \Ns}}
\DeclareMathOperator{\bigsqcuponJ}{\bigsqcup_{j\in J}}

\DeclareMathOperator{\bigcapon}{\bigcap_{k=0}^{n}}
\DeclareMathOperator{\bigcaponI}{\bigcap_{i\in I}}
\DeclareMathOperator{\bigcaponK}{\bigcap_{k\in K}}
\DeclareMathOperator{\bigcaponN}{\bigcap_{n\in \N}}
\DeclareMathOperator{\bigcaponNs}{\bigcap_{n\in \Ns}}
\DeclareMathOperator{\bigcaponJ}{\bigcap_{j\in J}}

\DeclareMathOperator{\A}{[a, +\infty[}
\renewcommand{\P}{\mathbb{P}}
\DeclareMathOperator{\Li}{L_i}
\DeclareMathOperator{\LoneI}{L^{1}(I)}

\DeclareMathOperator{\xtoinf}{\underset{x \to +\infty}{\longrightarrow}}
\DeclareMathOperator{\epsto}{\underset{\epsilon \to 0}{\longrightarrow}}
\DeclareMathOperator{\ntoinf}{\underset{n \to +\infty}{\longrightarrow}}
\DeclareMathOperator{\ptoinf}{\underset{p \to +\infty}{\longrightarrow}}
\DeclareMathOperator{\Ntoinf}{\underset{N \to +\infty}{\longrightarrow}}
\DeclareMathOperator{\ktoinf}{\underset{k \to +\infty}{\longrightarrow}}
\DeclareMathOperator{\limninf}{\underset{n \to +\infty}{\lim}}
\DeclareMathOperator{\limNinf}{\underset{N \to +\infty}{\lim}}
\DeclareMathOperator{\limkinf}{\underset{k \to +\infty}{\lim}}
\newcommand{\limp}[2]{\underset{#1 \to #2}{\lim}}

\DeclareMathOperator{\suponR}{\underset{x\in \R}{\sup}}
\DeclareMathOperator{\suponI}{\underset{x\in I}{\sup}}
\DeclareMathOperator{\suponK}{\underset{x \in K}{\sup}}
\newcommand{\supon}[1]{\underset{x \in #1}{\sup}}

\DeclareMathOperator{\infonR}{\underset{x\in \R}{\inf}}
\DeclareMathOperator{\infonI}{\underset{x\in I}{\inf}}
\DeclareMathOperator{\infonK}{\underset{x \in K}{\inf}}
\newcommand{\infon}[1]{\underset{x \in #1}{\inf}}

\DeclareMathOperator{\minonR}{\underset{x\in \R}{\min}}
\DeclareMathOperator{\minonI}{\underset{x\in I}{\min}}
\DeclareMathOperator{\minonK}{\underset{x \in K}{\min}}
\newcommand{\minon}[1]{\underset{x \in #1}{\min}}

\DeclareMathOperator{\maxonR}{\underset{x\in \R}{\max}}
\DeclareMathOperator{\maxonI}{\underset{x\in I}{\max}}
\DeclareMathOperator{\maxonK}{\underset{x \in K}{\max}}
\newcommand{\maxon}[1]{\underset{x \in #1}{\max}}

\newcommand{\cu}[1]{\overset{\text{C.U.}}{\underset{#1}{\xrightarrow{\hspace*{1cm}}}}}
\newcommand{\cs}[1]{\overset{\text{C.S.}}{\underset{#1}{\xrightarrow{\hspace*{1cm}}}}}
\newcommand{\cn}[1]{\overset{\text{C.N.}}{\underset{#1}{\xrightarrow{\hspace*{1cm}}}}}

\renewcommand{\listfigurename}{Liste de Figures}
\renewcommand{\contentsname}{Sommaire}


%%%%%%%%%%%%%%%%%%%%%%%%%%%%%%%%%%%%%%%%%%%%%%%%%%%%%%%%%%%%%%%%%%%%%%%%%%%%%%%
%                                                                             %
%                                                                             %
%                               Disposition                                   %
%                                                                             %
%                                                                             %
%%%%%%%%%%%%%%%%%%%%%%%%%%%%%%%%%%%%%%%%%%%%%%%%%%%%%%%%%%%%%%%%%%%%%%%%%%%%%%%

\newcommand{\mog}{\text{\og}}
\newcommand{\mfg}{\text{\fg}}

\let\dsp=\displaystyle
\let\tst=\textstyle
\let\spt=\scriptstyle

%% Quelques lettres grecques et autres caracteres plus jolis que la version
%% originale grace a AMS-LaTeX
\newcommand{\oldepsilon}{\epsilon}
\renewcommand{\epsilon}{\varepsilon}
\renewcommand{\emptyset}{\varnothing}
\renewcommand{\phi}{\varphi}
\renewcommand{\leq}{\leqslant}
\renewcommand{\geq}{\geqslant}
%
% \mathchardef\ordinarycolon\mathcode`\:
% \mathcode`\:=\string"8000
% \begingroup \catcode`\:=\active
%   \gdef:{\mathrel{\mathop\ordinarycolon}}
% \endgroup
%\mathtoolsset{centercolon}
\newcommand{\defeq}{\coloneqq}
\newcommand{\eqdef}{\eqqcolon}
%  \newcommand{\defeq}{=}
%  \newcommand{\eqdef}{=}
%%
%%
\newcommand{\annuleh}[1]{\setbox0=\hbox{$#1$}\ht0=0pt\box0}
\newcommand{\annulehp}[1]{\setbox0=\hbox{$#1$}\ht0=0pt\dp0=0pt\box0}

%%%%%%%%%%%%%%%%%%%%%%%%%%%%%%%%%%%%%%%%%%%%%%%%%%%%%%%%%%%%%%%%%%%%%%%%%%%%%%%
%                                                                             %
%                                                                             %
%                     Logique, ensembles, dénombrement                        %
%                                                                             %
%                                                                             %
%%%%%%%%%%%%%%%%%%%%%%%%%%%%%%%%%%%%%%%%%%%%%%%%%%%%%%%%%%%%%%%%%%%%%%%%%%%%%%%

% Quantificateurs
\newcommand{\qsep}{,\quad}

%% Logique
\newcommand{\et}{\quad\!\text{et}\quad\!}
\newcommand{\ou}{\quad\!\text{ou}\quad\!}
\newcommand{\non}{{\rm non}\ }
\newcommand{\implique}{\Longrightarrow}
\newcommand{\ssi}{\Longleftrightarrow}
\newcommand{\donc}{\text{donc}}
\newcommand{\rel}{\mathcal{R}}

%% Ensembles, opérations, applications, familles
\newcommand{\p}[1]{\left(#1\right)}
\newcommand{\cro}[1]{\left[#1\right]}
\newcommand{\ens}[1]{\left\{#1\right\}}
\DeclarePairedDelimiterX\set[1]\lbrace\rbrace{\def\given{\;\delimsize\vert\;}#1}
\newcommand{\enstq}[2]{\set*{#1\given #2}}
\newcommand{\ensim}[2]{\left\{#1 : #2\right\}}
\newcommand{\parties}[1]{\mathcal{P}\p{#1}}
\newcommand{\app}[3]{#1\colon #2 \rightarrow #3}
\newcommand{\appli}[5]{#1\colon #2 \rightarrow #3, #4 \mapsto #5}
\newcommand{\dspappli}[5]{\begin{array}[t]{lrcl}
\dsp{#1} : & \dsp{#2} & \longrightarrow & \dsp{#3} \\
    & \dsp{#4} & \longmapsto & \dsp{#5} \end{array}}
% \newcommand{\dspappli}[5]{#1\colon 
% \vtop{ \halign{ ##        &             ##        & ##                 \cr
%    \hfil $\dsp{#2}$ \hfil &   $\longrightarrow$   &$\dsp{#3}$ \hfil    \cr
%    \hfil $\dsp{#4}$ \hfil & $\longmapsto$         &\hfil$\dsp{#5}$\hfil\cr}}}
% \newcommand{\restri}[2]{#1|_{#2}}
\newcommand\restri[2]{{% we make the whole thing an ordinary symbol
  \left.\kern-\nulldelimiterspace % automatically resize the bar with \right
  #1 % the function
  \vphantom{\big|} % pretend it's a little taller at normal size
  \right|_{#2} % this is the delimiter
  }}
\newcommand\corestri[2]{{% we make the whole thing an ordinary symbol
  \left.\kern-\nulldelimiterspace % automatically resize the bar with \right
  #1 % the function
  \vphantom{\big|} % pretend it's a little taller at normal size
  \right|^{#2} % this is the delimiter
  }}
\newcommand\restricorestri[3]{{% we make the whole thing an ordinary symbol
  \left.\kern-\nulldelimiterspace % automatically resize the bar with \right
  #1 % the function
  \vphantom{\big|} % pretend it's a little taller at normal size
  \right|_{#2}^{#3} % this is the delimiter
  }}
\newcommand{\uple}[3]{\ensuremath{#1_{#2},\ldots,#1_{#3}}}
\newcommand{\nuple}[1]{\uple{#1}{1}{n}}
\newcommand{\imrec}[1]{f^{-1}}
\newcommand{\applirec}{{-1}} %% à enlever
%% \newcommand{\applirec}{{\textcircled{\tiny{-1}}}}


%% Dénombrement
\DeclareMathOperator{\card}{Card }
\DeclareMathOperator{\one}{\mathbb{1}}

%% Pour écrire des récurrences
\newenvironment{recurrence}
{%
  \begin{center}%
    $\mathcal{H}_n$~:\ \flqq
    \begin{varwidth}[t]{0.7\linewidth}
}{\end{itemize}}
\newcommand{\recinit}{\end{varwidth} \frqq\end{center}
  \begin{itemize}\item $\mathcal{H}_0$ est vraie ~: }
\newcommand{\rechere}{\item $\mathcal{H}_n \implique \mathcal{H}_{n+1}$~: }

%% Ensembles de nombres
\newcommand{\N}{\mathbb{N}}
\newcommand{\Ns}{{\N^*}}
\newcommand{\Z}{\mathbb{Z}}
\newcommand{\Zs}{\Z^*}
\newcommand{\Q}{\mathbb{Q}}
\newcommand{\Qs}{\Q^*}
\newcommand{\QP}{\Q_+}
\newcommand{\QPs}{\Q_+^*}
\newcommand{\QM}{\Q_-}
\newcommand{\QMs}{\Q_-^*}
\newcommand{\R}{\mathbb{R}}
\newcommand{\Rbar}{\overline{\R}}
\newcommand{\Rs}{{\R^*}}
\newcommand{\RP}{\R_+}
\newcommand{\RPs}{\R_+^*}
\newcommand{\RM}{\R_-}
\newcommand{\RMs}{\R_-^*}
\newcommand{\C}{\mathbb{C}}
\newcommand{\Cs}{\C^*}
\newcommand{\RouC}{\text{$\R$ (ou $\C$)}}
\newcommand{\CouR}{\text{$\C$ (ou $\R$)}}
\newcommand{\RbarouC}{\text{$\Rbar$ (ou $\C$)}}
\newcommand{\U}[1][!*!,!]
  {\mathbb{U}\ifthenelse{\equal{#1}{!*!,!}}{}{_{#1}}}
\newcommand{\Ha}{\mathbb{H}}

%%%%%%%%%%%%%%%%%%%%%%%%%%%%%%%%%%%%%%%%%%%%%%%%%%%%%%%%%%%%%%%%%%%%%%%%%%%%%%%
%                                                                             %
%                                                                             %
%                                  Algèbre                                    %
%                                                                             %
%                                                                             %
%%%%%%%%%%%%%%%%%%%%%%%%%%%%%%%%%%%%%%%%%%%%%%%%%%%%%%%%%%%%%%%%%%%%%%%%%%%%%%%

%% Entiers
\newcommand{\intere}[2]{\left\llbracket #1,#2\right\rrbracket}
\newcommand{\interefo}[2]{\left\llbracket #1,#2\right\llbracket}
\newcommand{\intereof}[2]{\left\rrbracket #1,#2\right\rrbracket}
\newcommand{\intereo}[2]{\left\rrbracket #1,#2\right\llbracket}

\newcommand{\kro}[2]{\delta_{#1,#2}}
\DeclareMathOperator{\pgcd}{pgcd}
\DeclareMathOperator{\ppcm}{ppcm}
\DeclarePairedDelimiter{\ceil}{\lceil}{\rceil}
\DeclarePairedDelimiter{\floor}{\lfloor}{\rfloor}
\newcommand{\ent}[1]{\floor*{#1}}
\newcommand{\ents}[1]{\ceil*{#1}}

%% Arithmétique
\newcommand{\egmodulo}[2]{\equiv #1\,\cro{#2}}
\newcommand{\modulo}[1]{\,\cro{#1}}

%% Systèmes linéaires
\def\syslin#1{\left\{
    \vcenter{\ialign{&$\hfil{}##{}$\crcr
                     #1\crcr}}
    \right.}
\makeatletter
\renewcommand*\env@matrix[1][*\c@MaxMatrixCols c]{%
  \hskip -\arraycolsep
  \let\@ifnextchar\new@ifnextchar
  \array{#1}}
\makeatother

%% Nombres complexes
\newcommand{\abs}[1]{\left|#1\right|}
\renewcommand{\Re}[1]{\mathop{\rm Re}#1}
\renewcommand{\Im}[1]{\mathop{\rm Im}#1}
\newcommand{\conj}[1]{\overline{#1}}
\newcommand{\sgn}{\mathop{\rm sgn}}
\newcommand{\ii}{{\rm i}}
\newcommand{\jj}{{\rm j}}
\newcommand{\e}{{\rm e}}

%% Structures
\newcommand{\K}{\ensuremath{\mathbb{K}}}
\newcommand{\KL}{\ensuremath{\mathbb{L}}}
\newcommand{\KM}{\ensuremath{\mathbb{M}}}
\newcommand{\Ks}{\ensuremath{\mathbb{K}^*}}
\newcommand{\zz}[1]{\ensuremath{\Z/{#1}\Z}}% Ensemble $Z/kZ$
\newcommand{\znz}{\ensuremath{\Z/n\Z}}% Abreviation pour $Z/nZ$
\newcommand{\gsym}[1]{\mathcal{S}_{#1}}
\newcommand{\galt}[1]{\mathcal{A}_{#1}}
\newcommand{\signat}[1]{\oldepsilon\p{#1}}

%% Polynômes, fractions rationnelles
\newcommand{\polyK}[1][!*!,!]{\K\ifthenelse{\equal{#1}{!*!,!}}{}{_{#1}}[X]}
\newcommand{\polyL}[1][!*!,!]{\KL\ifthenelse{\equal{#1}{!*!,!}}{}{_{#1}}[X]}
\newcommand{\polyZ}[1][!*!,!]{\Z\ifthenelse{\equal{#1}{!*!,!}}{}{_{#1}}[X]}
\newcommand{\polyQ}[1][!*!,!]{\Q\ifthenelse{\equal{#1}{!*!,!}}{}{_{#1}}[X]}
\newcommand{\polyR}[1][!*!,!]{\R\ifthenelse{\equal{#1}{!*!,!}}{}{_{#1}}[X]}
\newcommand{\polyC}[1][!*!,!]{\C\ifthenelse{\equal{#1}{!*!,!}}{}{_{#1}}[X]}
\newcommand{\fracK}[1][!*!,!]{\K\ifthenelse{\equal{#1}{!*!,!}}{}{_{#1}}(X)}
\newcommand{\fracQ}[1][!*!,!]{\Q\ifthenelse{\equal{#1}{!*!,!}}{}{_{#1}}(X)}
\newcommand{\fracR}[1][!*!,!]{\R\ifthenelse{\equal{#1}{!*!,!}}{}{_{#1}}(X)}
\newcommand{\fracC}[1][!*!,!]{\C\ifthenelse{\equal{#1}{!*!,!}}{}{_{#1}}(X)}
\newcommand{\poly}[1]{#1[X]}
\newcommand{\polyn}[2]{#1_{#2}[X]}
\newcommand{\fract}[1]{#1(X)}
\newcommand{\fractn}[2]{#1_{#2}(X)}
\newcommand{\Pmin}[1]{\pi_{#1}}

%% Algèbre linéaire
\newcommand{\Rev}{$\R$-espace vectoriel\xspace}
\newcommand{\Revs}{$\R$-espaces vectoriels\xspace}
\newcommand{\Cev}{$\C$-espace vectoriel\xspace}
\newcommand{\Cevs}{$\C$-espaces vectoriels\xspace}
\newcommand{\Kev}{$\K$-espace vectoriel\xspace}
\newcommand{\Kevs}{$\K$-espaces vectoriels\xspace}
\newcommand{\Lev}{$\KL$-espace vectoriel\xspace}
\newcommand{\Ralg}{$\R$-alg\`ebre\xspace}
\newcommand{\lin}[2]{\mathcal{L}\p{#1,#2}}
\newcommand{\Endo}[1]{\mathcal{L}\p{#1}}
\newcommand{\com}[1]{\mathcal{C}\p{#1}}
\renewcommand{\ker}{\mathop{\rm ker}}
\DeclareMathOperator{\im}{im}
\DeclareMathOperator{\rg}{rg}
\DeclareMathOperator{\codim}{codim}
\newcommand{\scal}[2]{\langle #1, #2 \rangle}
%% Matrices
\newcommand{\mat}[2]{\mathcal{M}_{#1}\p{#2}}
\newcommand{\gl}[2]{{\rm GL}_{#1}\p{#2}}
\newcommand{\glp}[1]{{\rm GL}_{#1}^+\p{\R}}
\newcommand{\glm}[1]{{\rm GL}_{#1}^-\p{\R}}
\DeclareMathOperator{\id}{Id}
%\newcommand{\trans}[1]{{#1}^{\rm T}}
\newcommand{\trans}[1]{{#1}^{\top}}
% \def\trans#1{\ifmmode\setbox9=\hbox{$#1$}\vphantom{\copy9}^t\!\box9%
%   \else\penalty1000\ #1\fi}
\DeclareMathOperator{\tr}{tr}
\renewcommand{\det}{\mathop{\rm det}}
\newcommand{\Com}{\mathop{\rm Com}}
\newcommand{\Sp}[1]{\mathop{\rm Sp}\left(#1\right)}
% \newcommand{\Sp}{\mathop{\rm Sp}}
\newcommand{\diag}[1]{\mathop{\rm{Diag}}\left(#1\right)}
\newcommand{\commutant}[1]{\mathop{\rm Com}\left(#1\right)}
\newcommand{\gdots}{\makebox[1em]{\raisebox{0em}{.}%
  \raisebox{0.5em}{.} \raisebox{1em}{.}}}
\newcommand{\Epp}{(E, +, \cdot)}

%%%%%%%%%%%%%%%%%%%%%%%%%%%%%%%%%%%%%%%%%%%%%%%%%%%%%%%%%%%%%%%%%%%%%%%%%%%%%%%
%                                                                             %
%                                                                             %
%                                Géométrie                                    %
%                                                                             %
%                                                                             %
%%%%%%%%%%%%%%%%%%%%%%%%%%%%%%%%%%%%%%%%%%%%%%%%%%%%%%%%%%%%%%%%%%%%%%%%%%%%%%%


%% Géométrie affine
\DeclareMathOperator{\bary}{Bar}
\newcommand{\ga}[1]{\mathop{\rm GA}\p{#1}}
\newcommand{\aff}[1]{\mathop{\rm Aff}\p{#1}}
\newcommand{\is}[1]{\mathop{\rm Is}\p{#1}}
\newcommand{\isp}[1]{\mathop{\rm Is}^+\p{#1}}

%% Géométrie euclidienne
\newcommand{\ve}[1]{\kern 2pt\overrightarrow{\kern -2pt#1}}
% \newcommand{\ps}[3][!*!,!]{
%   \ifthenelse{\equal{#1}{!*!,!}}{\left<#2|#3\right>}{}
%   \ifthenelse{\equal{#1}{0}}{<#2|#3>}{}}
\newcommand{\ps}[2]{\left\langle {#1}\middle|{#2}\right\rangle}
\newcommand{\norme}[1]{\left\|#1\right\|}
\newcommand{\normetriple}[1]{\vvvert \, #1 \, \vvvert}
\newcommand{\pmixte}[1]{\left[#1\right]}
\newcommand{\mesa}[1]{\overline{#1}}
\newcommand{\pvect}[2]{#1\wedge #2}
\newcommand{\Det}[2]{{\rm Det}\p{#1,#2}}
%%\newcommand{\Determinant}[1]{{\rm Det}\p{#1}}

%% Coordonnées
\newcommand{\coordp}[2]{\left|\begin{array}{l}\!\!#1 \\\!\!#2\end{array}\right.}
\newcommand{\coorde}[3]{\left|\begin{array}{l}\!\!#1 \\\!\!#2%
  \\\!\!#3\end{array}\right.}

%%%%%%%%%%%%%%%%%%%%%%%%%%%%%%%%%%%%%%%%%%%%%%%%%%%%%%%%%%%%%%%%%%%%%%%%%%%%%%%
%                                                                             %
%                                                                             %
%                                  Analyse                                    %
%                                                                             %
%                                                                             %
%%%%%%%%%%%%%%%%%%%%%%%%%%%%%%%%%%%%%%%%%%%%%%%%%%%%%%%%%%%%%%%%%%%%%%%%%%%%%%%

%% Fonctions
\newcommand{\dom}{\mathcal{D}}

%% Fonctions usuelles
% \DeclareMathOperator{\lg2}{log_2}
% \DeclareMathOperator{\lg10}{log_{10}}
% \DeclareMathOperator{\log2}{log_2}
% \DeclareMathOperator{\log10}{log_{10}}
\DeclareMathOperator{\cotan}{cotan}
\DeclareMathOperator{\ch}{ch}
\DeclareMathOperator{\sh}{sh}
%\DeclareMathOperator{\coth}{coth}
\DeclareMathOperator{\magtanh}{th}
  \let\tanh=\magtanh
  \let\th=\magtanh
\DeclareMathOperator{\magarcsin}{Arcsin}
  \let\arcsin=\magarcsin
\DeclareMathOperator{\magarctan}{Arctan}
  \let\arctan=\magarctan
\DeclareMathOperator{\magarccos}{Arccos}
  \let\arccos=\magarccos
\DeclareMathOperator{\arccotan}{Arccotan}
%%\DeclareMathOperator{\coth}{coth}
\DeclareMathOperator{\argch}{Argch}
\DeclareMathOperator{\argsh}{Argsh}
\DeclareMathOperator{\argth}{Argth}
\DeclareMathOperator{\argcoth}{Argcoth}
\DeclareMathOperator{\Arg}{Arg}
\DeclareMathOperator{\Ln}{Ln}

%% Parties de R
\newcommand{\Rb}{\overline{\R}}
\newcommand{\interf}[2]{\left[#1,#2\right]}
\newcommand{\intero}[2]{\left]#1,#2\right[}
\newcommand{\interfo}[2]{\left[#1,#2\right[}
\newcommand{\interof}[2]{\left]#1,#2\right]}
\newcommand{\intergf}[2]{\text{\og $\left[#1,#2\right]$ \fg}}
\newcommand{\intergo}[2]{\text{\og $\left]#1,#2\right[$ \fg}}
\newcommand{\intergfo}[2]{\text{\og $\left[#1,#2\right[$ \fg}}
\newcommand{\intergof}[2]{\text{\og $\left]#1,#2\right]$ \fg}}

\renewcommand{\O}{{\rm O}}
%% Limites
\newcommand{\tend}{\xrightarrow{}}
\newcommand{\tendvers}[2]{\xrightarrow[#1\rightarrow #2]{}}
\newcommand{\tendversd}[2]{\xrightarrow[
                            \substack{#1\rightarrow #2 \\ #1\geq #2}]{}}
\newcommand{\tendversg}[2]{\xrightarrow[
                            \substack{#1\rightarrow #2 \\ #1\leq #2}]{}}
\newcommand{\tendversdp}[2]{\xrightarrow[
                            \substack{#1\rightarrow #2 \\ #1>#2}]{}}
\newcommand{\tendversgp}[2]{\xrightarrow[
                            \substack{#1\rightarrow #2 \\ #1<#2}]{}}
\newcommand{\tendversp}[2]{\xrightarrow[
                           \substack{#1\rightarrow #2 \\ #1\not=#2}]{}}
\DeclareMathOperator{\DL}{DL}
\DeclareMathOperator{\DA}{DA}
\newcommand{\petito}[3]{\underset{#1 \rightarrow #2}{\text{o}}\left(#3\right)}
\newcommand{\petitop}[3]{\underset{\substack{#1 \rightarrow #2\\#1\neq #2}}{\text{o}}%
  \p{#3}}
\newcommand{\petitog}[3]{\underset{\substack{#1 \rightarrow #2\\#1\leq #2}}{\text{o}}%
  \p{#3}}
\newcommand{\petitod}[3]{\underset{\substack{#1 \rightarrow #2\\#1\geq #2}}{\text{o}}%
  \p{#3}}
\newcommand{\petitogp}[3]{\underset{\substack{#1 \rightarrow #2\\#1< #2}}{\text{o}}%
  \p{#3}}
\newcommand{\petitodp}[3]{\underset{\substack{#1 \rightarrow #2\\#1> #2}}{\text{o}}%
  \p{#3}}
\newcommand{\petitozero}[2]{\underset{#1 \rightarrow 0}{\text{o}}\left(#2\right)}
\newcommand{\petitozerop}[2]{\underset{\substack{#1 \rightarrow 0\\#1\neq 0}}{\text{o}}%
  \p{#2}}
\newcommand{\petitozerog}[2]{\underset{\substack{#1 \rightarrow 0\\#1\leq 0}}{\text{o}}%
  \p{#2}}
\newcommand{\petitozerod}[2]{\underset{\substack{#1 \rightarrow 0\\#1\geq 0}}{\text{o}}%
  \p{#2}}
\newcommand{\petitozerogp}[2]{\underset{\substack{#1 \rightarrow 0\\#1< 0}}{\text{o}}%
  \p{#2}}
\newcommand{\petitozerodp}[2]{\underset{\substack{#1 \rightarrow 0\\#1> 0}}{\text{o}}%
  \p{#2}}
\newcommand{\petitov}[4]{\underset{\substack{#1 \rightarrow #3\\#1 #2 #3}}{\text{o}}%
  \left(#4\right)}
\newcommand{\petitoinfty}[2]{\underset{#1 \rightarrow +\infty}{\text{o}}\left(#2\right)}

%\newcommand{\grandtheta}[3]{\underset{#1 \rightarrow #2}{\Theta\left(#3\right)}

\newcommand{\grando}[3]{\underset{#1 \rightarrow #2}{\text{O}}\left(#3\right)}
\newcommand{\grandop}[3]{\underset{\substack{#1 \rightarrow #2\\#1\neq #2}}{\text{O}}%
  \p{#3}}
\newcommand{\grandog}[3]{\underset{\substack{#1 \rightarrow #2\\#1\leq #2}}{\text{O}}%
  \p{#3}}
\newcommand{\grandod}[3]{\underset{\substack{#1 \rightarrow #2\\#1\geq #2}}{\text{O}}%
  \p{#3}}
\newcommand{\grandogp}[3]{\underset{\substack{#1 \rightarrow #2\\#1< #2}}{\text{O}}%
  \p{#3}}
\newcommand{\grandodp}[3]{\underset{\substack{#1 \rightarrow #2\\#1> #2}}{\text{O}}%
  \p{#3}}
\newcommand{\grandozero}[2]{\underset{#1 \rightarrow 0}{\text{O}}\left(#2\right)}
\newcommand{\grandozerop}[2]{\underset{\substack{#1 \rightarrow 0\\#1\neq 0}}{\text{O}}%
  \p{#2}}
\newcommand{\grandozerog}[2]{\underset{\substack{#1 \rightarrow 0\\#1\leq 0}}{\text{O}}%
  \p{#2}}
\newcommand{\grandozerod}[2]{\underset{\substack{#1 \rightarrow 0\\#1\geq 0}}{\text{O}}%
  \p{#2}}
\newcommand{\grandozerogp}[2]{\underset{\substack{#1 \rightarrow 0\\#1< 0}}{\text{O}}%
  \p{#2}}
\newcommand{\grandozerodp}[2]{\underset{\substack{#1 \rightarrow 0\\#1> 0}}{\text{O}}%
  \p{#2}}

\newcommand{\equi}[2]{\underset{#1\rightarrow #2}{\sim}}
\newcommand{\equip}[2]{\underset{\substack{#1\rightarrow #2\\#1\neq #2}}{\sim}}
\newcommand{\equid}[2]{\underset{\substack{#1\rightarrow #2\\#1\geq #2}}{\sim}}
\newcommand{\equig}[2]{\underset{\substack{#1\rightarrow #2\\#1\leq #2}}{\sim}}
\newcommand{\equidp}[2]{\underset{\substack{#1\rightarrow #2\\#1>#2}}{\sim}}
\newcommand{\equigp}[2]{\underset{\substack{#1\rightarrow #2\\#1<#2}}{\sim}}

%% Suites
\newcommand{\suite}[3]{\p{#1_#2}_{#2\in#3}}
\newcommand{\suiten}[1]{\p{#1_n}_{n\in\N}}
\newcommand{\suitej}[1]{\p{#1_j}_{j\in J}}
\newcommand{\suitek}[1]{\p{#1_k}_{k\in K}}
\newcommand{\equiS}{\underset{n\rightarrow+\infty}{\sim}}

%% Dérivation
\newcommand{\derd}[2]{\dfrac{\text{d}#1}{\text{d}#2}}
\newcommand{\classed}[1]{\mathscr{D}^{#1}}
\newcommand{\classec}[1]{\mathscr{C}^{#1}}
\newcommand{\classecinf}{\classec{\infty}}
\newcommand{\fracd}[2]{\ffrac{{\rm d}#1}{{\rm d}#2}}

%% Intégration, primitives
\newcommand{\integ}[4]{\displaystyle\int_{#1}^{#2} #3\,\text{d}#4}
\newcommand{\integinv}[4]{\displaystyle\int_{#1}^{#2} \dfrac{\text{d}#4}{#3}}
\newcommand{\integppdi}[5]{\displaystyle\int_{#1}^{#2} \underbrace{#3}_{\text{d\'erive}}
                           \overbrace{#4}^{\text{int\`egre}}\,\text{d}#5}
\newcommand{\integppid}[5]{\displaystyle\int_{#1}^{#2} \overbrace{#3}^{\text{int\`egre}}
                           \underbrace{#4}_{\text{d\'erive}}\,\text{d}#5}
\newcommand{\evaldiff}[3]{\left[ #1 \right]_{#2}^{#3}}
\newcommand{\prim}[2]{\integ{}{}{#1}{#2}}
\newcommand{\priminv}[2]{\integinv{}{}{#1}{#2}}
\newcommand{\primppdi}[3]{\integppdi{}{}{#1}{#2}{#3}}
\newcommand{\primppid}[3]{\integppid{}{}{#1}{#2}{#3}}

%% Fonctions de plusieurs variables
\newcommand{\parfrac}[2]{\dfrac{\partial #1}{\partial #2}}
\newcommand{\integd}[4]{\int\!\!\!\!\int_{#1} #2\,\text{d}#3\text{d}#4}
\newcommand{\integdinv}[4]{\int\!\!\!\!\int_{#1}
   \frac{\text{d}#3\text{d}#4}{#2}}

%% Topologie
\def\build#1_#2^#3{\mathrel{\mathop{\kern0pt#1}\limits_{#2}^{#3}}}
\newcommand{\rond}[1]{\build#1_{}^{\>\circ}}
\newcommand{\interieur}[1]{\build#1_{}^{\>\circ}}
\newcommand{\dis}[2]{{\rm d}\!\p{#1,#2}}
\newcommand{\tendversnorme}[1]{\overset{\norm{\;}_{#1}}{\underset{n \to +\infty}{\xrightarrow{\hspace*{1cm}}}}}


%%%%%%%%%%%%%%%%%%%%%%%%%%%%%%%%%%%%%%%%%%%%%%%%%%%%%%%%%%%%%%%%%%%%%%%%%%%%%%%
%                                                                             %
%                                                                             %
%                                   Autres                                    %
%                                                                             %
%                                                                             %
%%%%%%%%%%%%%%%%%%%%%%%%%%%%%%%%%%%%%%%%%%%%%%%%%%%%%%%%%%%%%%%%%%%%%%%%%%%%%%%

\def\og{\leavevmode\raise.3ex\hbox{$\sst\langle\!\langle$}}
\def\fg{\leavevmode\raise.3ex\hbox{$\sst\,\rangle\!\rangle$}}
\newcommand{\cc}{\mathcal{C}}
\newcommand{\cd}{\mathcal{D}}
\newcommand{\ce}{\mathcal{E}}
\newcommand{\cf}{\mathcal{F}}
\newcommand{\cg}{\mathcal{G}}

%%%%%%%%%%%%%%%%%%%%%%%%%%%%%%%%%%%%%%%%%%%%%%%%%%%%%%%%%%%%%%%%%%%%%%%%%%%%%%%
%                                                                             %
%                                                                             %
%                                  Probabilités                               %
%                                                                             %
%                                                                             %
%%%%%%%%%%%%%%%%%%%%%%%%%%%%%%%%%%%%%%%%%%%%%%%%%%%%%%%%%%%%%%%%%%%%%%%%%%%%%%%
\newcommand{\comp}[1]{\overline{#1}}
\newcommand{\pr}[1]{\ifthenelse{\equal{#1}{}}{\mathbf{P}}{\mathbf{P}\p{#1}}}
\newcommand{\prcro}[1]{\mathbf{P}\cro{#1}}
\newcommand{\prc}[2]{\mathbf{P}\p{#1|#2}}
\newcommand{\prcp}[2]{\ifthenelse{\equal{#1}{}}{\mathbf{P}_{#2}}{\mathbf{P}_{#2}\p{#1}}}
\newcommand{\prob}{\mathbb{P}}
% \newcommand{\Proba}[1]{\mathbb{P}\p{#1}}
% \newcommand{\Probac}[2]{\mathbb{P}_{#2}\p{#1}}
\newcommand{\Esp}[1]{\mathbb{E}\p{#1}}
\newcommand{\Var}[1]{\mathbb{V}\p{#1}}
\newcommand{\Etype}[1]{\sigma\p{#1}}
%\newcommand{\cov}[2]{\mathrm{Cov}\p{#1,#2}}
\newcommand{\cor}[2]{\rho\p{#1,#2}}

\DeclareMathOperator{\At}{\mathcal{A}}
\DeclareMathOperator{\LBinom}{\mathcal{B}}
\DeclareMathOperator{\LGeo}{\mathcal{G}}
\DeclareMathOperator{\LUnif}{\mathcal{U}}
\DeclareMathOperator{\LPois}{\mathcal{P}}

\newcommand{\esp}[1]{E \left( #1 \right)}   %espéranc
\newcommand{\var}[1]{V \left( #1 \right)}   %variance
\newcommand{\ect}[1]{\sigma \left( #1 \right)}   %ecart-type
\newcommand{\cov}[1]{{\rm Cov} \left( #1 \right)}   %covariance

\newcommand{\norm}[1]{\left\| #1 \right\|}      %norme
\newcommand{\OAP}{(\Omega, \At, \P)}

%%%%%%%%%%%%%%%%%%%%%%%%%%%%%%%%%%%%%%%%%%%%%%%%%%%%%%%%%%%%%%%%%%%%%%%%%%%%%%%
%                                                                             %
%                                                                             %
%                                  Informatique                               %
%                                                                             %
%                                                                             %
%%%%%%%%%%%%%%%%%%%%%%%%%%%%%%%%%%%%%%%%%%%%%%%%%%%%%%%%%%%%%%%%%%%%%%%%%%%%%%%
\newcommand{\matlab}{Matlab\xspace}


%%%%%%%%%%%%%%%%%%%%%%%%%%%%%%%%%%%%%%%%%%%%%%%%%%%%%%%%%%%%%%%%%%%%%%%%%%%%%%%
%                                                                             %
%                                                                             %
%                                  Walter Appel                               %
%                                                                             %
%                                                                             %
%%%%%%%%%%%%%%%%%%%%%%%%%%%%%%%%%%%%%%%%%%%%%%%%%%%%%%%%%%%%%%%%%%%%%%%%%%%%%%%
% \Esp \waEsp
% \Var \waVar
% \petito \wapetito
% \mat \wamat
% \tendvers \watendvers
% \newenvironment{exo}{\section{Exercice}}{}
%\newenvironment{ListeMaths}{\begin{itemize}}{\end{itemize}}
% \newif\if@MORE
% \def\ListeM@thStyle#1{\textbf{\textit{#1)}}}
% \def\ListeM@thItemStyle#1{\roman{#1}}
% \newenvironment{ListeMaths}{%
% 	\renewcommand{\theenumi}{\ListeM@thItemStyle{enumi}}%
% 	\renewcommand{\labelenumi}{\qquad\ListeM@thStyle{{\theenumi}}}%
% 	\renewcommand{\theenumii}{\alph{enumii}}%
% 	\renewcommand{\labelenumii}{\quad\ListeM@thStyle{\theenumii}}%
% 	\renewcommand{\labelenumiii}{\ListeM@thStyle{\theenumiii}}%
% 	\renewcommand{\p@enumii}{}\begin{enumerate}}{\end{enumerate}}
%   %\renewcommand{\labelenumi}{\qquad\ListeM@thStyle{{\theenumi}}}
%   %\renewcommand{\theenumii}{\alph{enumii}}
%   %\renewcommand{\labelenumii}{\quad\ListeM@thStyle{\theenumii}}
%   %\renewcommand{\labelenumiii}{\ListeM@thStyle{\theenumiii}}
%   %\renewcommand{\p@enumii}{}
% %}

% \newcommand{\eps}{\varepsilon}
%\newcommand{\dfrac}[2]{\frac{\partial #1}{\partial #2}}


% \DeclareMathOperator*{\mypetito}{o}
% \DeclareMathOperator*{\mygrando}{O}
% \newcommand{\wapetito}[1][]{\ifthenelse{\equal{#1}{}}{\mypetito\limits}
%   {{\mypetito\limits_{#1}}}}
% \DeclareMathOperator{\Sym}{Sym}
% \DeclareMathOperator*{\mySim}{\sim}
% \newcommand{\Sim}{\mySim\limits}
% \newcommand{\dsum}{\ensuremath{\sum\limits}}
% \DeclareMathOperator{\Id}{Id}
% \newcommand{\mathgras}[1]{\ensuremath{%
%         {\text{\mathversion{bold}\ensuremath{#1}}}%
%         }}
% \newcommand{\textegras}[1]{\textbf{#1}}%
% %\newcommand{\textegras}[1]{\grastrue\textbf{\mathversion{bold}#1}\grasfalse}%
% \newcommand{\gras}{\grastrue\bfseries\mathversion{bold}}%
% \newcommand{\nongras}{\grasfalse\mathversion{normal}\normalfont}
% \newcommand{\suitepaire}[2][n]{({#2}_{2#1})_{#1\in\N}}
% \newcommand{\suiteimpaire}[2][n]{({#2}_{2#1+1})_{#1\in\N}}
% \newcommand{\suiteabs}[2][n]{\bpar{\abs{{#2}_{#1}}}_{#1\in\N}}
% \newcommand{\suitez}[2][n]{({#2}_{#1})_{#1\in\Z}}
% \newcommand{\suiteZ}[2][n]{({#2}_{#1})_{#1\in\Z}}
% \newcommand{\Suite}[1]{({#1}_{n})_{n\in{\N^*}}}
% \newcommand{\ssuite}[2][\phi]{({#2}_{#1(n)})_{n\in\N}}



% \newlength{\restsubwidth}
% \newlength{\restsubheight}
% \newlength{\restsubmoreheight}
% \setlength{\restsubmoreheight}{4 pt}
% \newcommand{\rest}[2]{%
%         \settowidth{\restsubwidth}{\ensuremath{{}_{#2}}}
%         \settoheight{\restsubheight}{\ensuremath{{}_{#2}}}
%         \addtolength{\restsubheight}{\restsubmoreheight}
%         \ensuremath{{#1\hskip 0.5 pt}_{\vrule\kern2pt\parbox[b][%
%         \the\restsubheight][b]{\the\restsubwidth}{%
%                         \ensuremath{{}_{#2}}}}}
%         }


% \newcommand{\f}[2]{{\ensuremath{\mathchoice%
%         {\dfrac{#1}{#2}}
%         {\dfrac{#1}{#2}}
%         {\frac{#1}{#2}}
%         {\frac{#1}{#2}}
%         }}}

% %\newcommand{\pa}[1]{\ensuremath{\left(#1\right)}}
% \newcommand{\pa}[2][9]{%
%         \ifthenelse{#1 = 0}
%                 {\ensuremath{#2}}{}%
%         \ifthenelse{#1 = 1}
%                 {\ensuremath{(#2)}}{}%
%         \ifthenelse{#1 = 2}
%                 {\ensuremath{\big(#2\big)}}{}%
%         \ifthenelse{#1 = 3}
%                 {\ensuremath{\Big(#2\Big)}}{}%
%         \ifthenelse{#1 = 4}
%                 {\ensuremath{\bigg(#2\bigg)}}{}%
%         \ifthenelse{#1 = 5}
%                 {\ensuremath{\Bigg(#2\Bigg)}}{}%
%         \ifthenelse{#1 = 9}
%                 {\ensuremath{\left(#2\right)}}{}%
% }



% \newcommand{\paf}[2]{\ensuremath{\left(\f{#1}{#2}\right)}}
% \newcommand{\absf}[2]{\ensuremath{\left|\f{#1}{#2}\right|}}
% \newcommand{\bpar}[1]{\ensuremath{%
%     \mathchoice%
%     {\big(#1\big)}
%     {\big(#1\big)}
%     {{\textstyle(}#1{\textstyle)}}
%     {{\textstyle(}#1{\textstyle)}}
% }}
% \newcommand{\Bpar}[1]{\ensuremath{%
%     \mathchoice%
%     {\Big(#1\Big)}
%     {\Big(#1\Big)}
%     {{\textstyle(}#1{\textstyle)}}
%     {{\textstyle(}#1{\textstyle)}}
% }}
% \newcommand{\bbpar}[1]{\ensuremath{%
%     \mathchoice%
%     {\bigg(#1\bigg)}
%     {\bigg(#1\bigg)}
%     {{\textstyle(}#1{\textstyle)}}
%     {{\textstyle(}#1{\textstyle)}}
% }}


% % On peut aussi vouloir des crochets de la bonne taille plutôt que des 
% % parenthèses (analogue de \pa: mêmes recommandations).
% % Exemple: $\crochets{ \f{1}{2} x \ln x }$
% %\newcommand{\crochets}[1]{\ensuremath{\left[#1\right]}}
% \newcommand{\crochets}[2][9]{%
%         \ifthenelse{#1 = 0}
%                 {\ensuremath{#2}}{}%
%         \ifthenelse{#1 = 1}
%                 {\ensuremath{[#2]}}{}%
%         \ifthenelse{#1 = 2}
%                 {\ensuremath{\big[#2\big]}}{}%
%         \ifthenelse{#1 = 3}
%                 {\ensuremath{\Big[#2\Big]}}{}%
%         \ifthenelse{#1 = 4}
%                 {\ensuremath{\bigg[#2\bigg]}}{}%
%         \ifthenelse{#1 = 5}
%                 {\ensuremath{\Bigg[#2\Bigg]}}{}%
%         \ifthenelse{#1 = 9}
%                 {\ensuremath{\left[#2\right]}}{}%
% }
% %\newcommand{\pac}[1]{\ensuremath{\crochets{#1}}}                % Synonyme
% \let\pac\crochets
% \newcommand{\bpac}[1]{\ensuremath{\big[{#1}\big]}}
% \newcommand{\Bpac}[1]{\ensuremath{\Big[{#1}\Big]}}

% % Ou pourquoi pas des accolades (analogue de \pa: mêmes recommandations).
% \newcommand{\accolades}[2][9]{%
%         \ifthenelse{#1 = 0}
%                 {\ensuremath{#2}}{}%
%         \ifthenelse{#1 = 1}
%                 {\ensuremath{\{#2\}}}{}%
%         \ifthenelse{#1 = 2}
%                 {\ensuremath{\big\{#2\big\}}}{}%
%         \ifthenelse{#1 = 3}
%                 {\ensuremath{\Big\{#2\Big\}}}{}%
%         \ifthenelse{#1 = 4}
%                 {\ensuremath{\bigg\{#2\bigg\}}}{}%
%         \ifthenelse{#1 = 5}
%                 {\ensuremath{\Bigg\{#2\Bigg\}}}{}%
%         \ifthenelse{#1 = 9}
%                 {\ensuremath{\left\{#2\right\}}}{}%
% }
% \newcommand{\paa}[2][9]{\ensuremath{\accolades[#1]{#2}}}                % Synonyme
% %\newcommand{\accolades}[1]{\ensuremath{\left\{#1\right\}}}
% \newcommand{\bpaa}[1]{\ensuremath{\big\{{#1}\big\}}}
% \newcommand{\Bpaa}[1]{\ensuremath{\Big\{{#1}\Big\}}}
% %\newcommand{\paa}[1]{\ensuremath{\accolades{#1}}}               % Synonyme


% \newcommand{\Int}[2]{\ensuremath{\mathchoice%
%         {\displaystyle\int_{#1}^{#2}}
%         {\displaystyle\int_{#1}^{#2}}
%         {\int_{#1}^{#2}}
%         {\int_{#1}^{#2}}
%         }}
% \newcommand{\Sum}[2]{\ensuremath{\sum\limits}_{#1}^{#2}}
% % \newcommand{\dsum}{\ensuremath{\sum\limits}}
% \newcommand{\dlim}{\ensuremath{\lim\limits}}
% \newcommand{\dint}{\ensuremath{\displaystyle\int}}
% %%% Pour les intervalles.
% \newcommand{\MonSeparateur}{\,;\,}

% \makeatletter
% \newcommand{\d@wn}[1]{\ifthenelse{\equal{#1}{}}{-\infty}{#1}}
% \newcommand{\@pper}[1]{\ifthenelse{\equal{#1}{}}{+\infty}{#1}}
% \newcommand*{\intff}{\@ifstar\@integerintff\@realintff}
% \newcommand*{\@integerintff}[2]{%
%   \intn{#1}{#2}}%\ensuremath{[\![\d@wn{#1}\MonSeparateur\@pper{#2}]\!]}}
% \newcommand*{\@realintff}[2]{%
%   \ensuremath{\left[\,\d@wn{#1}\MonSeparateur\@pper{#2}\,\right]}}
%         % Intervalle fermé à gauche et à droite.
%         % Exemple : la fonction sinus est une bijection de
%         % $\intff{0}{\f{\pi}{2}}$ sur $\intff{0}{1}$\,. La forme étoilée
%         % donne un intervalle d'entiers, la forme non étoilée un intervalle
%         % de réels.
% \newcommand{\intoo}[2]{%
%   \ensuremath{\left]\,\d@wn{#1}\MonSeparateur\@pper{#2}\,\right[}}
%         % Intervalle ouvert à gauche et à droite.
%         % Exemple : La fonction $x \mapsto \f{1}{x}$ est décroissante sur 
%         % $\intoo{\minf}{0}$ et sur $\intoo{0}{\pinf}$\,, mais pas sur leur
%         % réunion.
% \newcommand{\intof}[2]{%
%   \ensuremath{\left]\,\d@wn{#1}\MonSeparateur\@pper{#2}\,\right]}}
%         % Intervalle ouvert à gauche et fermé à droite.
% \newcommand*{\intfo}{\@ifstar\@integerintfo\@realintfo}
% \newcommand*{\@integerintfo}[2]{%
%   \ensuremath{[\![\d@wn{#1}\MonSeparateur\@pper{#2}[\![}}
% \newcommand*{\@realintfo}[2]{%
%   \ensuremath{\left[\,\d@wn{#1}\MonSeparateur\@pper{#2}\,\right[}}
%         % Intervalle fermé à gauche et ouvert à droite.

% \newcommand{\intn}[2]{%
%   \ensuremath{%
%     \left[\!\left[\,#1\MonSeparateur\,#2\,\right]\!\right]}}

% \newlength{\haut}
% \renewcommand{\intn}[2]{\ensuremath{
%     \settowidth{\haut}{$\left[\makebox[0mm][c]{$#1#2$}\right.$}   %Pour calculer la
%     \left[\hspace{-.5\haut}\left[#1\MonSeparateur#2\right]%
%       \hspace{-0.5\haut}\right]}}%
% \makeatother
% % :%s/\\petito/\\wapetito/gc



% %--------------------------- Les \centers ----------------------------

% \newcounter{vcenterstest}
% \newlength{\leftlength}
% \newlength{\rightlength}
% \newlength{\vcentersskip}
% \newlength{\vcentersskipbelow}
% \newcommand{\leftcentersright}[4][1]{%
%         \settowidth{\leftlength}{#2}%
%         \settowidth{\rightlength}{#4}%
%         \setcounter{vcenterstest}{#1}%
%         \ifthenelse{\value{vcenterstest} < 0}%
%                 {\setlength{\vcentersskip}{-\medskipamount}}{}%
%         \ifthenelse{\value{vcenterstest} = 0}%
%                 {\setlength{\vcentersskip}{0pt}}{}%
%         \ifthenelse{\value{vcenterstest} = 1}%
%                 {\setlength{\vcentersskip}{\smallskipamount}}{}%
%         \ifthenelse{\value{vcenterstest} = 2}%
%                 {\setlength{\vcentersskip}{\medskipamount}}{}%
%         \ifthenelse{\value{vcenterstest} = 3}%
%                 {\setlength{\vcentersskip}{\bigskipamount}}{}%
%         \ifthenelse{\value{vcenterstest} = 4}%
%                 {\setlength{\vcentersskip}{1cm}}{}%
%                 % On laisse un espace vertical défini par l'argument
%                 % optionnel #1
%         \vskip\vcentersskip%
%                 % On place #2 et on recule de sa longueur
%         \noindent#2\hskip-\leftlength%%
%                 % On centre #3
%         \hfill#3\hfill%
%                 % On va au bout de la ligne, on recule de la longueur de #4 et
%                 % on place #4
%         \mbox{}\hskip-\rightlength#4%
%                 % On laisse un espace vertical défini par l'argument
%                 % optionnel #1
%         \ifthenelse{\value{vcenterstest}>0}%
%         {\vskip\vcentersskip}{\vskip\smallskipamount}\noindent%
%         \ignorespaces}
                
% % Mettre un commentaire à gauche et centrer le deuxième argument obligatoire
% % Exemple : \leftcenters{par suite}{$x_0 = 7$}
% \newcommand{\leftcenters}[3][2]{\leftcentersright[#1]{#2}{#3}{}}
% \newcommand{\noleftcenters}[3][2]{#2\leftcentersright[#1]{}{#3}{}}

% % Pour la compatibilité ascendante. N'en tenez pas compte.
% \newcommand{\remandcenters}[3][2]{\leftcentersright[#1]{#2}{#3}{}}
                
% % Centrer et mettre à droite le deuxième argument obligatoire
% % Exemple : \centersright{$x_0 = 7$}{\deux}.
% \newcommand{\centersright}[3][2]{\leftcentersright[#1]{}{#2}{#3}}
                
% % Centrer seulement
% % Exemple : \centers{Coucou !}
% \newcommand{\centers}[2][2]{\leftcentersright[#1]{}{#2}{}}

% % Un autre nom pour la même chose
% \newcommand{\centrer}[2][2]{\leftcentersright[#1]{}{#2}{}}

% \newcommand{\leftencadre}[3][2]{\leftcentersright[#1]{#2}{\fbox{#3}}{}}
% \newcommand{\noleftencadre}[3][2]{#2\leftcentersright[#1]{}{\fbox{#3}}{}}
% \newcommand{\encadreright}[3][2]{\leftcentersright[#1]{}{\fbox{#2}}{#3}}
% \newcommand{\encadre}[2][2]{\leftcentersright[#1]{}{\fbox{#2}}{}}
% \newcommand{\leftencadreright}[4][2]{\leftcentersright[#1]{#2}{\fbox{#3}}{#4}}
% \newcommand{\tencadre}[2][2]{{\fbox{#2}}}


% \newcommand{\donccenters}[1]{\leftcenters{donc}{#1}}
% \newcommand{\doucenters}[1]{\leftcenters{d'où}{#1}}
% \newcommand{\carcenters}[1]{\leftcenters{car}{#1}}
% \newcommand{\cadcenters}[1]{\leftcenters{c'est-à-dire}{#1}}
% \newcommand{\puiscenters}[1]{\leftcenters{puis}{#1}}
% \newcommand{\ainsicenters}[1]{\leftcenters{Ainsi,}{#1}}
% \newcommand{\depluscenters}[1]{\leftcenters{De plus,}{#1}}
% \newcommand{\aveccenters}[1]{\leftcenters{avec}{#1}}
% \newcommand{\soitcenters}[1]{\leftcenters{soit}{#1}}
% \newcommand{\alorscenters}[1]{\leftcenters{alors}{#1}}
% \newcommand{\Alorscenters}[1]{\leftcenters{Alors}{#1}}
% \newcommand{\maiscenters}[1]{\leftcenters{mais}{#1}}
% \newcommand{\commecenters}[1]{\leftcenters{Comme}{#1}}
% \newcommand{\ailleurscenters}[1]{\leftcenters{Par ailleurs,}{#1}}
% \newcommand{\orcenters}[1]{\leftcenters{Or}{#1}}
% \newcommand{\etcenters}[1]{\leftcenters{et}{#1}}
% \newcommand{\deduitcenters}[1]{\leftcenters{On en déduit}{#1}}
% \newcommand{\parsuitecenters}[1]{\leftcenters{Par suite,}{#1}}
% \newcommand{\parcsqcenters}[1]{\leftcenters{Par conséquent,}{#1}}
% \newcommand{\ilvientcenters}[1]{\leftcenters{Il vient}{#1}}
% \newcommand{\conclusioncenters}[1]{\leftcenters{Conclusion :}{#1}}
% \newcommand{\finalementcenters}[1]{\leftcenters{Finalement :}{#1}}

%                                 % Une équation avec un nom
%                                 % \encadrenom{$\laplacien\phi=0$}
%                                 % {équation de Poisson}
% \newcommand{\encadrenom}[2]{\encadreright{#1}{\textbf{\textup{(#2)}}}}
% \newcommand{\centersnom}[2]{\centersright{#1}{\textbf{\textup{(#2)}}}}

%                                 % Les versions numerotées : attention à
%                                 % l'ordre 
% %% Exemple :
% %% \nomcentersnum[$*$]{$\laplacien\phi=0$}{équation de Poisson}
% \newcommand{\nomcentersnum}[3][7]{%
%   \refstepcounter{equation}%
%   \leftcentersright{\textbf{\textup{(#3)}}}%
%   {#2}{(\theequation)}%
%   \ifthenelse{\equal{#1}{7}}{}{\tag{#1}}}
% \newcommand{\centersnum}[1]{%
%   \refstepcounter{equation}%
%   \centersright{#1}{(\theequation)}}%
  

% \newcommand{\encadrenum}[1]{%
%   \refstepcounter{equation}\encadreright{#1}{(\theequation)}}
% \newcommand{\leftencadrenum}[3][7]{%
%   \refstepcounter{equation}%
%   \leftencadreright{#2}{#3}{(\theequation)}%
%   \ifthenelse{\equal{#1}{7}}{}{\tag{#1}}}
% \newcommand{\leftcentersnum}[3][7]{%
%   \refstepcounter{equation}%
%   \leftcentersright{#2}{#3}{(\theequation)}%
%   \ifthenelse{\equal{#1}{7}}{}{\tag{#1}}}


% %%%%% J'en ai marre de mettre des _{\text{blah}} partout.
% % Très pratique en physique et en chimie, où il y a beaucoup d'indices.
% % Par exemple : $T\stext{liquéfaction} = 0 °C$. 
% \newcommand{\stext}[1]{\ensuremath{{}_{\text{#1}}}}

% %%%%% Tant qu'on y est, ras-le-bol d'utiliser des touches compliquées:
% \newcommand{\moinsun}{\ensuremath{{}^{-1}}}
% \newcommand{\moinsdeux}{\ensuremath{{}^{-2}}}
% \newcommand{\moinstrois}{\ensuremath{{}^{-3}}}

% %%DefBCPST.tex

% \newcommand{\MySmallCaps}[1]{\begingroup\fontfamily{cmr}\fontseries{m}
% \fontshape{sc}\selectfont#1\endgroup}


% % \newcommand{\cfexo}[1]{\textbf{\emph{Cf. exercice~\vref{#1}.}}}
% % \newcommand{\cfexos}[1]{\textbf{\emph{Cf. exercices~#1}.}}
% \newcommand{\dejavu}[2]{\textbf{\emph{Déjà donné en~\elz{#1},
%       exercice~\vref{#2}}}} 

% % \newcommand{\Defi}{\medskip\centerline{--- \fbox{\textbf{Et le défi du
% %         jour}} ---}}
% \newcommand{\Defi}{}
% \newenvironment{HP}{\DUR}{\FACILE}


% \newcommand{\newcolumn}{\vfil\vskip300cm}
% % \newcommand{\fondnoir}[1]{\psframebox[fillstyle=solid,fillcolor=black,linecolor=white,linewidth=0mm]{\white#1}}

% % \newcommand{\fondgris}[1]{\psframebox[fillstyle=solid,fillcolor=lightgray,linecolor=white,linewidth=0mm]{#1}}
% % \newcommand{\fondgrad}[1]{\psframebox[fillstyle=gradient,gradbegin=lightgray,gradend=darkgray,linecolor=white,linewidth=0mm]{{\white#1}}}


% % Pour les exos de Maple/Matlab/Scilab
% \providecommand{\Maple}{\textsf{Maple}\xspace}
% \newcommand{\MAPLE}{\index{Maple@\texttt{Maple}}(Avec \textsf{Maple})\par}
% \providecommand{\maplegras}[1]{%
%   \begingroup\fontfamily{pcr}\fontseries{b}\selectfont#1\endgroup}
% \providecommand{\Matlab}{\textsf{Matlab}\xspace}
% \newcommand{\MATLAB}{\index{Matlab@\texttt{Matlab}}(Avec \textsf{Matlab})\par}
% \providecommand{\Scilab}{\textsf{Scilab}\xspace}
% \providecommand{\SciPad}{\textsf{SciPad}\xspace}
% \newcommand{\SCILAB}{\index{Scilab@\texttt{Scilab}}(Avec \textsf{Scilab})\par}

% % Pour mettre les repères temporels
% \newcommand{\Repere}[2]{}

% \newcommand{\REPERE}[1]{}

% \providecommand{\VoF}{\noindent\qquad$\square$ \texttt{Vrai}\qquad $\square$
%                         \texttt{Faux}\par}


% \newcommand{\asterix}{ \medskip\par \begingroup\centering\Huge\ding{167}
%         \\*[6mm]\endgroup}

% %\newcommand{\diff}[1]{($#1$)\ }
% \newcommand{\diff}[1]{(\ifthenelse{\equal{#1}{\circ}}{$\star$}{%
%     \ifthenelse{\equal{#1}{*}}{$\star$}{%
%       \ifthenelse{\equal{#1}{**}}{$\star\,\star$}{%
%         \ifthenelse{\equal{#1}{****}}{$\star\!\star\!\star\!\star$}{%
%           \ifthenelse{\equal{#1}{***}}{$\star\!\star\!\star$}%
%           {\textbf{???}}}}}})%
%   \xspace}

% \newcommand{\ex}{\mathrm{e}}
% \newcommand{\defeg}{\stackrel{\scriptscriptstyle\textrm{\textup{d\'ef.}}}{=}}
%                                 % Si jamais je veux finalement mettre une
%                                 % fraction « en ligne » 
% \newcommand{\nofrac}[2]{{#1}/{#2}}
%                                 % si le premier argument exige des
%                                 % parenthèses 
% \newcommand{\pnofrac}[2]{{(#1)}/{#2}}
%                                 % et si le deuxième argument exige des
%                                 % parenthèses... 
% \newcommand{\nofracp}[2]{{#1}/{(#2)}}
%                                 % et enfin si le tout requière ces
%                                 % parenthèses... 
% \newcommand{\nopfrac}[2]{\pa{#1/#2}}

% \newcommand{\explique}[1]{\tag*{\makebox[0mm][r]{#1}\ }}
% %%% Plus et moins l'infini.
% \newcommand{\plus}{\mbox{\raisebox{.2mm}{\tiny{\ensuremath{+}}}}}
% \newcommand{\moins}{\mbox{\raisebox{.2mm}{\tiny{\ensuremath{-}}}}}
%         % \plus et \moins ne servent qu'en indice et dans \pinf et \minf,
%         % pas dans les additions et soustractions normales.
% \newcommand{\pinf}{+\infty}
% \newcommand{\minf}{-\infty}
% \def\qqavecqq{\qquad\text{avec}\qquad}
% \def\et{\text{ et }}
% \def\qetq{\quad\text{et}\quad}
% \def\qqetqq{\qquad\text{et}\qquad}
% \def\ou{\text{ ou }}
% \def\qouq{\quad\text{ou}\quad}
% \def\qqouqq{\qquad\text{ou}\qquad}
% % \newcommand{\pa}[1]{\p{#1}}
% \newcommand{\zb}{\bar{z}}
% \newcommand{\dd}{\mathrm{d}}%\ifgras\boldmath{d}\else\mathrm{d}\fi}
% %\newcommand{\dz}{\dd z}
% \newcommand{\dx}{\dd x}
% \newcommand{\du}{\dd u}
% \newcommand{\dy}{\dd y}
% %\newcommand{\dr}{\dd r}
% \newcommand{\dzb}{\dd\hspace{.4pt}\zb} 
% \newcommand{\ddz}{\dep{\phantom{z}}{z}}
% \newcommand{\ddt}{\dep{\phantom{t}}{t}}
% \newcommand{\ddx}{\dep{\phantom{x}}{x}}
% \newcommand{\ddy}{\dep{\phantom{y}}{y}}
% \newcommand{\ddzb}{\dep{\phantom{z}}{\,\zb}}
% \newcommand{\dt}{\dd t}
% \newcommand{\ds}{\dd s}
% \newcommand{\DD}{\mathrm{D}}


% %********** Macros dépendant de la présentation en 1/2 colonnes *******
% \newcounter{NbList}
% \setcounter{NbList}{1}
% \newcommand{\NombreColonnes}{1}
% %% Permet, au sein d'un exercice, de demander une présentation en plusieurs
% %% colonnes (1,2,3...)
% \newcommand{\NbLM}[2]{%
%   \if\NombreColonnes1\setcounter{NbList}#1\else\setcounter{NbList}#2\fi}

%                                 % Ensuite, on fait
%                                 % \begin{ListeMaths}[\value{NbList}]



%                                 % Si on est avec 2 colonnes, on veut faire
%                                 % plus de sauts de lignes... 
% \newcommand{\sautcond}[1][]{%
%   \ifthenelse{\equal{\NombreColonnes}{1}}%
%                                 % En une colonne, on ne fait rien 
%   {#1}% 
%                                 % En deux colonnes, on saute à la ligne
%   {\notag\\}% 
% }

% \newcommand{\sautcondet}[1][]{%
%   \ifthenelse{\equal{\NombreColonnes}{1}}%
%                                 % En une colonne, on ne fait rien 
%   {#1}% 
%                                 % En deux colonnes, on saute à la ligne
%   {\notag\\ & }% 
% }

% \newcommand{\etsautcond}{%
%   \ifthenelse{\equal{\NombreColonnes}{1}}%
%                                 % En une colonne, on fait un « & »
%   {&}% 
%                                 % En deux colonnes, on saute à la ligne
%   {\notag\\}% 
% }

% \newcommand{\centerscond}[2][]{%
%   \ifthenelse{\equal{\NombreColonnes}{1}}%
%                                 %  En une colonne, on ne fait rien d'autre
%                                 %  que l'argument [optionnel], par exemple
%                                 %  un \quad.
%   {#1#2}%
%                                 %  En deux colonnes, on centre sur une
%                                 %  ligne à part 
%   {\centers{#2}}%
% }

%                                 % Fait une équation multine si deux
%                                 % colonnes, et rien sinon.
% \newcommand{\multcond}[3][]{%
%   \ifthenelse{\equal{\NombreColonnes}{1}}%
%                                 % En une colonne, tout d'affilée, avec
%                                 % l'argument optionnel (un \quad ou un
%                                 % \qquad en général) entre les deux
%                                 % argument obligatoires.
%   {\centers{#2#1#3}}%
%                                 % En deux colonnes, plus subtil : on émule
%                                 % un {multline} en serrant la première
%                                 % ligne à gauche et la seconde ligne à
%                                 % droite. 
%   {\par\noindent\begin{minipage}{\linewidth}\leftcenters{\quad#2}{}\vspace*{-4mm}%
%     \centersright{}{#3\quad\makebox[0mm]{}}\end{minipage}}%
% }

%                                 % Un ordre conditionnel à deux colonnes,
%                                 % avec l'argument [optionnel] pour une
%                                 % colonne. 
% \newcommand{\cond}[2][]{%
%   \ifthenelse{\equal{\NombreColonnes}{1}}%
%                                 % En une colonne, on fait l'ordre 1
%   {#1}% 
%                                 % En deux colonnes, on fait l'ordre 2
%   {#2}% 
% }

% \newcommand{\absurde}[1]{$\blacktriangleright$~\textegras{#1}}
% \newcommand{\surdab}{~$\blacktriangleleft$}
% \newcommand{\Arctan}{\arctan}
% \newcommand{\Argth}{\argth}
% \newcommand{\Arccos}{\arccos}

% \DeclareMathOperator{\Card}{Card}
% \newcommand{\tq}{\;;\;}
% \newcommand{\donne}{\mathchoice{\longmapsto}{\mapsto}{\mapsto}{\mapsto}}
% \newcommand{\numero}{}

% \newcommand{\ProbasShape}[1]{\text{\fontshape{n}\fontseries{bx}\selectfont\sffamily #1}}
% \newcommand{\Parties}{\mathcal{P}}
% \newcommand{\Ps}{\ProbasShape{P}}
% \renewcommand{\Pr}{\Ps}
% \newcommand{\waEsp}{\ProbasShape{E}}
% \newcommand{\Loi}{\hookrightarrow}
% \DeclareMathOperator{\Cov}{\ProbasShape{Cov}}
% \DeclareMathOperator{\waVar}{\ProbasShape{V}}%{Var}

% \newcommand{\toprule}{\hline}
% \newcommand{\midrule}{\hline}
% \newcommand{\bottomrule}{\hline}
% \newenvironment{waremarque}{}{}
% \newenvironment{Systeme}{%
%   \ensuremath{\left\{
%       \begin{aligned}}}
%       {\end{aligned}\right.}
% \newcommand{\dans}{\mathchoice{\longrightarrow}{\rightarrow}{\rightarrow}{\rightarrow}} 
% \newcommand{\dansC}{\dans\C}
% \newcommand{\dansR}{\dans\R}
% \newcommand{\nabs}[1]{\lvert #1 \rvert}
% \newcommand{\babs}[1]{\big\lvert#1\big\rvert}
% \newcommand{\Babs}[1]{\Big\lvert#1\Big\rvert}
% \newcommand{\bbabs}[1]{\bigg\lvert#1\bigg\rvert}

% \newcommand{\sujet}[1]{}%
% \newcommand{\ecole}[1]{}%
% \newcommand{\annee}[1]{}%
% \newcommand{\taupin}[1]{}%
% \newcommand{\filiere}[1]{}%
% \newcommand{\cfexo}[1]{}%
% \newcommand{\cfexos}[1]{}%

% \newcommand{\pap}[1]{\cro{#1}}

% \newcommand{\Comb}[3][f]{\ensuremath{%
%     \ifthenelse{\equal{#1}{f}}% 
%     {{C}_{#2}^{#3}}%               pour les français
%     {\binom{#2}{#3}} % pour les anglais
%     }}
% \newcommand{\comb}[3][a]{\ensuremath{%
%     \ifthenelse{\equal{#1}{a}}% 
%     {\binom{#2}{#3}} % pour les anglais
%     {{C}_{#2}^{#3}}%               pour les français
%     }}

% \newlength{\calculskip}
% \newcommand{\calculvskip}[1]{% 
%   \ifthenelse{#1 = 0}{\setlength{\calculskip}{0pt}}{}%
%   \ifthenelse{#1 = 1}{\setlength{\calculskip}{\smallskipamount}}{}%
%   \ifthenelse{#1 = 2}{\setlength{\calculskip}{\medskipamount}}{}%
%   \ifthenelse{#1 = 3}{\setlength{\calculskip}{\bigskipamount}}{}%
%   \ifthenelse{#1 = 4}{\setlength{\calculskip}{1cm}}{}%
%   \vskip\calculskip
% }

% \newcommand{\methode}[1]{\par\bigskip
%   \hfill{\textbf{%
%       \ifthenelse{\equal{#1}{bourrin}}{Méthode bourrine}{%
%         \ifthenelse{\equal{#1}{astuce}}{Méthode astucieuse}{%
%           \ifthenelse{\equal{#1}{PC}}{Méthode PC}{%
%             \ifthenelse{\equal{#1}{MP}}{Méthode MP}{%
%               \ifcase#1 Autre \or Première \or Deuxième \or
%               Troisième \or Autre \fi{} méthode}}}}}}\hfill\null\\*[1em]%
%       \ignorespaces}

% \newcommand{\sachant}{|}
% \newcommand{\textdef}[1]{\textbf{#1}}

% % Celui utilisé en pratique (admet en option le nombre de colonnes) et
% % n'utilise que l'option (a)
% \makeatletter
% \newenvironment{Compare}[1][a]{%
%   \ifthenelse{\equal{#1}{a}}
%   {\renewcommand{\theenumi}{\textit{\alph{enumi}}}%
%     \renewcommand{\labelenumi}{\qquad{\textbf{(\theenumi)}}}
%     \renewcommand{\theenumii}{\textit{\alph{enumii}}}%
%     \renewcommand{\labelenumii}{\qquad{\textbf{(\theenumii)}}}}
%   {\ifthenelse{\equal{#1}{i}}
%     {\renewcommand{\theenumi}{\textbf{\textit{\roman{enumi})}}}%
%       \renewcommand{\labelenumi}{\qquad{\theenumi}}
%     \renewcommand{\theenumii}{\textbf{\textit{\roman{enumii})}}}%
%       \renewcommand{\labelenumii}{\qquad{\theenumii}}
%   \renewcommand{\p@enumii}{}}}
% {\renewcommand{\theenumi}{\textbf{\arabic{enumi}}}%
%   \renewcommand{\labelenumi}{\qquad{(\theenumi)}}
%   \renewcommand{\theenumii}{\textbf{\arabic{enumii}}}%
%   \renewcommand{\labelenumii}{\qquad{(\theenumii)}}}
% \begin{enumerate}}{\end{enumerate}}
% \makeatother

% \newenvironment{Equ}{%
%   \begin{Compare}[a]}{\end{Compare}}
% \newcounter{TestMulti}%
% \newenvironment{CompareCol}[1][1]{
%   \ifthenelse{\equal{#1}{1}}{\global\setcounter{TestMulti}{0}}
%   {\global\setcounter{TestMulti}{1}\begin{multicols}{#1}}%
%     \begin{Compare}[a]}
%     {\end{Compare}
%     \ifthenelse{\equal{\value{TestMulti}}{0}}{}{\end{multicols}}}
% \newcommand{\Ssi}{si et seulement si\xspace}
% \newcommand{\EQU}{il y a équivalence entre les énoncés~:}
% \newenvironment{Equiv}[1][1]{\EQU\begin{CompareCol}[#1]}{\end{CompareCol}}
% \newenvironment{EQUIV}[1][1]{Il y a équivalence entre les
%   énoncés~:\begin{CompareCol}[#1]}{\end{CompareCol}}


% \newcommand{\FonctionHK}[6][n]{%
%   \ifthenelse{\equal{#1}{s}}%
%   % Option «serré»
%   {\ensuremath{#2\colon%
%     \raisebox{.8ex}{\mbox{\begin{array}[t]{%|
%                                              r@{\ }c@{\ }l}%
%        \scriptstyle#3
%       & \scriptstyle\longrightarrow & \scriptstyle#4 \\
%       \scriptstyle #5 & \scriptstyle\longmapsto & \scriptstyle #6 \\
%     \end{array}}}%
%   }}
%   % Option «normal»
%   {\ensuremath{#2\colon%
%     \begin{array}[t]{%|
%                        r@{\ }c@{\ }l}%
%        #3
%       & \longrightarrow & #4 \\[2mm]
%       #5 & \longmapsto & \displaystyle #6 \\
%     \end{array}%
%   }}%
% }

% \providecommand{\Fonction}[6][n]{
%   \centers{$\FonctionHK[#1]{#2}{#3}{#4}{#5}{#6}$}
% }

%                                 % Fais une fonction « dans le texte ».
% \providecommand{\fonction}[5]{%
%   $\displaystyle #1:\,#2\longrightarrow#3,\quad#4\longmapsto#5$}
% \newcommand{\Ker}{\ker}

% \newcommand{\petitcarre}[1][]{\mbox{\Pisymbol{pzd}{111}\;%
%   \ifthenelse{\equal{#1}{}}{}{\textegras{#1}}}}
% \newcommand{\AC}{\mathcal{A}}
% \newcommand{\BC}{\mathcal{B}}
% \newcommand{\CC}{\mathcal{C}}
% \newcommand{\EC}{\mathcal{E}}
% \newcommand{\FC}{\mathcal{F}}
% \newcommand{\UC}{\mathcal{U}}
% \newcommand{\XC}{\mathcal{X}}
% \newcommand{\can}{\text{\texttt{can}}}
% \newcommand{\BCb}{\overline{\BC}}
% \newcommand{\Boule}[3][]{\BC_{#1}\left(#2\,;\,#3\right)}
% \newcommand{\BouleO}[3][]{\BC_{#1}(#2\,;\,#3)}
% \newcommand{\BouleB}[3][]{\BCb_{#1}(#2\,;\,#3)}
% \newcommand{\BouleF}[3][]{\BCb_{#1}(#2\,;\,#3)}


% \DeclareMathOperator{\END}{\textsf{END}}
% \DeclareMathOperator{\APPL}{\textsf{APPL}}
% \DeclareMathOperator{\MAT}{\textsf{MAT}}

% \newcommand{\base}[2][n]{(#2_1,\dots,#2_{#1})}
% \newcommand{\based}[2][n]{(#2^{*}_1,\dots,#2^{*}_#1)}
%                                 % Les bases polynomiales commencent souvent
%                                 % à 0.
% \newcommand{\basep}[2][n]{(#2_0,\dots,#2_{#1})}

% \newcommand{\liste}[2][n]{#2_1,\dots,#2_{#1}}
%                                 % Les bases polynomiales commencent souvent
%                                 % à 0.
% \newcommand{\listep}[2][n]{#2_0,\dots,#2_{#1}}


% \newcommand{\gG}{\mathfrak{g}}
% \newcommand{\TG}{\mathfrak{T}}
% \newcommand{\POCA}[1]{\textsf{{PoCa}}(#1)}

%                                 % Polynôme caractéristique d'une matrice
%                                 % (majuscule) ! 
% \newcommand{\Poca}[1]{\chi_{_{#1}}}
%                                 % Polynôme caractéristique d'un
%                                 % endomorphisme (minuscule)
% \newcommand{\poca}[1]{\chi_{#1}}
% \newcommand{\PoCa}[1]{\POCA{#1}}

% \newenvironment{indic}{\par\hfill\begin{minipage}{.9\linewidth}%
%     \small\medskip\noindent\textsc{{Indication : }}\itshape%
%   }{%
%   \end{minipage}%
%   \medskip\normalfont\par}
	
% \newenvironment{lemme}{}{}
% \newenvironment{demo}{}{}

% \newcommand{\ii}{{\mathrm{i}}}
% % 
% % fonctions de classe C0, C1, C2, Cinfini
% \newcommand{\classe}{\mathscr{C}}
% \newcommand{\Cont}{\classe}
% \newcommand{\Czero}{\classe^{0}}
% \newcommand{\Cun}{\classe^{1}}
% \newcommand{\Cdeux}{\classe^{2}}
% \newcommand{\Ctrois}{\classe^{3}}
% \newcommand{\Cinf}{\classe^{\infty}}
% \newcommand{\Cn}{\classe^{n}}
% \newcommand{\Ck}{\classe^{k}}

% \newcommand{\EnonceColumnBreak}{}
% \newcommand{\AQT}{}
% \newcommand{\analyse}{Analyse}
% \newcommand{\synthese}{Synthèse}
% % 
% % \DeclareMathOperator{\p@@rplus}{\stackrel{\perp}{\oplus}}
% % \newcommand{\perplus}{\p@@rplus\limits}
% % \DeclareMathOperator{\bigp@@rplus}{\stackrel{\perp}{\bigoplus}}
% % \newcommand{\bigperplus}{\bigp@@rplus\limits} 
% % 
% \DeclareMathOperator{\Vect}{Vect}
% \DeclareMathOperator{\Un}{\mathbf{1}}
% \DeclareMathOperator{\Aut}{Aut}
% \DeclareMathOperator{\antidiag}{antiDiag}
% \DeclareMathOperator{\Antidiag}{AntiDiag}

% \newenvironment{bulitemize}{\begin{itemize}}{\end{itemize}}

% \newcommand{\LS}{\mathscr{L}}
% \newcommand{\LE}{\LS(E)}

% \newcommand{\LEF}{\LS(E,F)}
% \newcommand{\Cab}{\ensuremath{\classe\bpar{\intff{a}{b}}}}
% \newcommand{\CabR}{\ensuremath{\classe\bpar{\intff{a}{b},\R}}}
% \newcommand{\CabC}{\ensuremath{\classe\bpar{\intff{a}{b},\C}}}
% \newcommand{\CabK}{\ensuremath{\classe\bpar{\intff{a}{b},\K}}}
% \newcommand{\Cmab}{\ensuremath{\classe_m\bpar{\intff{a}{b}}}}
% \newcommand{\CmabC}{\ensuremath{\classe_m\bpar{\intff{a}{b},\C}}}
% \newcommand{\CmabK}{\ensuremath{\classe_m\bpar{\intff{a}{b},\K}}}
% \newcommand{\CmabE}{\ensuremath{\classe_m\bpar{\intff{a}{b},E}}}
% \newcommand{\CabE}{\ensuremath{\classe\bpar{\intff{a}{b},E}}}
% \newcommand{\Cmir}{\classe_m(I,\R)}
% \newcommand{\Cmirp}{\classe_m(I,\R^+)}
% \newcommand{\Cmik}{\classe_m(I,\K)}
% \newcommand{\Cmic}{\classe_m(I,\C)}
% \newcommand{\Cdp}{\classe_{[2\pi]}}
% \newcommand{\Cmdp}{\classe^{\text{m}}_{[2\pi]}}
% \newcommand{\Cpdp}{\classe'_{[2\pi]}}
% \newcommand{\Pp}{{\cal P}}
% \newcommand{\rV}{\vect{r}}
% \newcommand{\pV}{\vect{p}}
% \newcommand{\qV}{\vect{q}}
% \newcommand{\kV}{\vect{k}}
% \newcommand{\hV}{\vect{h}}
% \newcommand{\AV}{\vect{A}}
% \newcommand{\XV}{\vect{X}}
% \newcommand{\xV}{\vect{x}}
% \newcommand{\vV}{\vect{v}}
% \newcommand{\aV}{\vect{a}}
% \newcommand{\yV}{\vect{y}}
% \newcommand{\zV}{\vect{z}}
% \newcommand{\wV}{\vect{w}}
% \newcommand{\uV}{\vect{u}}
% \newcommand{\YV}{\vect{Y}}
% %\newcommand{\yV}{\vect{y}}
% \newcommand{\iV}{\vect{i}}
% \newcommand{\jV}{\vect{j}}
% \newcommand{\eV}{\vect{e}}
% \newcommand{\PV}{\vect{P}}
% \newcommand{\RV}{\vect{R}}
% \newcommand{\eQ}{\qv{e}}
% \newcommand{\uQ}{\qv{u}}
% \newcommand{\vQ}{\qv{v}}
% \newcommand{\jQ}{\qv{j}}
% \newcommand{\XQ}{\qv{X}}
% \newcommand{\AQ}{\qv{A}}
% \newcommand{\xQ}{\qv{x}}
% \newcommand{\omegaQ}{\vect{\omega}}
% \newcommand{\sigmaQ}{\vect{\sigma}}
% \newcommand{\alphaQ}{\vect{\alpha}}
% \newcommand{\betaQ}{\vect{\beta}}
% \newcommand{\ASV}{\pmb{\mathscr{A}}}
% \newcommand{\WS}{\mathscr{W}}
% \providecommand{\sub}[1]{_{\text{#1}}}
% \providecommand{\super}[1]{^{\text{#1}}}
% \providecommand{\subR}{_{\text{\textsc{(r)}}}}
% \providecommand{\subI}{_{\text{\textsc{(i)}}}}
% \newcommand{\submax}{_{\text{max}}}
% \newcommand{\submin}{_{\text{min}}}
% \newcommand{\moy}[1]{\left\langle #1\right\rangle}
%                                 % Sous-groupe monogène engendré par~$a$.
% \newcommand{\mono}[1]{\left\langle#1\right\rangle}

% \newcommand{\Lu}{\ensuremath{L^1}}
% \newcommand{\Ld}{\ensuremath{L^2}}


% \DeclareMathOperator{\Img}{Im}

% \DeclareMathOperator{\Ind}{Ind}
% \DeclareMathOperator*{\wamat}{mat}
\DeclareMathOperator{\supp}{Supp}
% % \DeclareMathOperator{\cotan}{cotan}
% \DeclareMathOperator{\zero}{Zero}

% \newcommand{\ES}{\mathscr{E}}
% \newcommand{\FS}{\mathscr{F}}
% \newcommand{\GS}{\mathscr{G}}
% \newcommand{\GL}{\mathrm{GL}}
% \newcommand{\Gl}{\mathscr{G}\!\ell}
% \newcommand{\GLS}{\Gl}
% \newcommand{\GlS}{\Gl}
% \newcommand{\SlS}{\mathscr{S}\!\ell}
% \newcommand{\Sl}{\SlS}
% \newcommand{\GLn}{\GL_n}
% \newcommand{\GLE}{\GLS(E)}
% \newcommand{\GlE}{\GlS(E)}
% \newcommand{\GLnR}{\GL_n(\R)}
% \newcommand{\GLnC}{\GL_n(\C)}
% \newcommand{\GLnK}{\GL_n(\K)}
% \DeclareMathOperator{\SL}{SL}
% \newcommand{\Mat}{\mathfrak{M}}
% \newcommand{\MnR}{\Mat_n(\R)}
% \newcommand{\MnC}{\Mat_n(\C)}
% \newcommand{\MnK}{\Mat_n(\K)}
% \newcommand{\MnpK}{\Mat_{np}(\K)}
% \newcommand{\OnR}{\OR_n(\R)}
% \newcommand{\SOnR}{\SO_n(\R)}

% \newcommand{\watendvers}[1]{\xrightarrow[{#1}]{}}% Plus le truc
% \newcommand{\THa}{\stackrel{\mbox{\tiny T.H.}}\longrightarrow}
% \newcommand{\CVU}{\xrightarrow{\textsc{c.v.u.}}}
% \newcommand{\CVS}{\xrightarrow{\textsc{c.v.s.}}}

% \newcommand{\resp}{respectivement\xspace}

% \DeclareMathOperator{\wadiag}{diag}
% \newcommand{\jj}{{\mathrm{j}}}
% \newcommand{\jjb}{\bar{\jj}}

% \newcommand{\lscaldel}{\langle}
% \newcommand{\rscaldel}{\rangle}
% \newlength{\ScalHeight}\newlength{\ScalDepth}
% \newcommand{\scal}[2]{%
%   \settoheight{\ScalHeight}{$\displaystyle #1#2$}%
%   \settodepth{\ScalDepth}{$\displaystyle #1#2$}%
%   \addtolength{\ScalHeight}{\ScalDepth}%
%   \left\lscaldel%
%     \rule[-\ScalDepth]{0mm}{\ScalHeight}%
%     #1%
%   \right|%
%   \left.%
%     \rule[-\ScalDepth]{0mm}{\ScalHeight}%
%     \!\! #2%
%   \right\rscaldel%
% }

% \newcommand{\nscal}[2]{\lscaldel#1\big|#2\rscaldel}
%                                 % Pour faire (f(x)|y) : \bscal{f(x)}{y}
%                                 % Ainsi, les parenthèses sont en \big
%                                 % d'office... 
% \newcommand{\bscal}[2]{\big\lscaldel#1\big|#2\big\rscaldel}
% \newcommand{\Bscal}[2]{\Big\lscaldel#1\Big|#2\Big\rscaldel}

% \newcommand{\ket}[1]{\left|#1\right\rangle}
% \newcommand{\bra}[1]{\left\langle#1\right|}
% \newcommand{\bracket}[2]{%
%   \settoheight{\ScalHeight}{$\displaystyle #1#2$}%
%   \settodepth{\ScalDepth}{$\displaystyle #1#2$}%
%   \addtolength{\ScalHeight}{\ScalDepth}%
%   \left\langle%
%     \rule[-\ScalDepth]{0mm}{\ScalHeight}%
%     #1\right|\left. %
%     \rule[-\ScalDepth]{0mm}{\ScalHeight}%
%     \! #2\right\rangle
% }

% \newcommand{\bbracket}[2]{\big\langle#1\big|#2\big\rangle}
% \newcommand{\Bbracket}[2]{\Big\langle#1\Big|#2\Big\rangle}
% \newcommand{\nbracket}[2]{\langle#1\big|#2\rangle}

% \newcommand{\asurb}[2]{{\raisebox{+.2em}{$\scriptstyle #1$} \!
%         \raisebox{+.2ex}{$\scriptstyle/$}\!
%         \raisebox{-.1ex}{$\scriptstyle #2$} } }

% \DeclareMathOperator{\Cte}{%
%   {\text{\upshape C}^{\text{\underline{\upshape te}}}}} 

% \newcommand{\sur}[2]{\mathchoice%
%   {\raisebox{.5ex}{$#1$}/\raisebox{-.25ex}{$#2$}}
%   {\raisebox{.5ex}{$#1$}/\raisebox{-.25ex}{$#2$}}
%   {\raisebox{.3ex}{$\scriptstyle#1$}\scriptstyle
%     \!/\!\raisebox{-.15ex}{$\scriptstyle#2$}}
%   {\raisebox{.3ex}{$\scriptstyle#1$}\scriptstyle
%     \!/\!\raisebox{-.15ex}{$\scriptstyle#2$}}}
		
% \newenvironment{syst}[1][r]%
%         {\ensuremath{\left \{ \hskip -1.5 mm \begin{array}{#1@{\ =\ }l}}}%
%         {\end{array}\right.}
% \newenvironment{systsimple}[1][r]%
%         {\ensuremath{\begin{array}[t]{#1@{\ =\ }l}}}%
%         {\end{array}}
% \newenvironment{varsyst}[1][r]%
%         {\ensuremath{\left \{ \hskip -1.5 mm%
%          \begin{array}{#1@{\ }c@{\ }l}}}%
%         {\end{array}\right.}
% \newenvironment{accolade}[1][r]%
%         {\ensuremath{\left \{ \hskip -1.5 mm \begin{array}{#1@{\quad}l}}}%
%         {\end{array}\right.}
% \newenvironment{leqsystsimple}[1][r]%
%         {\ensuremath{\begin{array}[t]{#1@{\ \leq\ }l}}}%
%         {\end{array}}
% \newenvironment{rcl}[1][r]%
%         {\ensuremath{\begin{array}[t]{#1@{\ }c@{\ }l}}}%
%         {\end{array}}

% \newcommand{\limt}{\ensuremath{\lim\limits}}

% \renewcommand{\iint}{\int\!\!\!\int}
% \renewcommand{\iiint}{\int\!\!\!\int\!\!\!\int}

% \newcommand{\scr}{\mathscr}

% \newcommand{\dep}[2]{\mathchoice%
%   {\dfrac{\displaystyle\partial #1}{\displaystyle\partial #2}}
%   {\dfrac{\displaystyle\partial #1}{\displaystyle\partial #2}}
%   {\frac{\displaystyle\partial #1}{\displaystyle\partial #2}}
%   {\frac{\displaystyle\partial #1}{\displaystyle\partial #2}}
%   }
%                                 % ... et d'ordre 2
% \newcommand{\depd}[3]{\dfrac{\displaystyle\partial^2 #1}{\displaystyle
%     \ifthenelse{\equal{#2}{#3}}{\partial#2^2}{\partial
%         #2\,\partial#3}}}
				
% \newcommand{\laplacien}{\bigtriangleup}
% \DeclareMathOperator{\grad}{\text{\textrm{\textbf{grad}}}}

% \newcommand{\Ent}[1]{E\p{#1}}
% \newcommand{\Normale}{\mathscr{N}}
% \newcommand{\FRN}{\Phi}%{\mathfrak{N}}
% \newcommand{\FDN}{\phi}%{\mathfrak{n}}
% \newcommand{\AFINIR}{}

% \newcommand{\pv}{\;;\;} % Point-virgule
% \newcommand{\ect}{\mathgras{\sigma}}

% \let\pap\crochets
% \let\bpap\bpac
% \let\Bpap\Bpac

% \newcommand{\ordre}[1]{%  Un ordre simple, dans le texte
%   {\upshape\texttt{\NoAutoSpaceBeforeFDP#1\AutoSpaceBeforeFDP}}}


\newcommand{\cpp}{{{C\nolinebreak[4]\hspace{-.05em}\raisebox{.4ex}{\tiny\bf ++}}}\xspace}